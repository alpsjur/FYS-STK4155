\section{Introduction}
\label{sec:introduction}

Artifical neural networks (NNs) offer a flexible way of performing data analysis, and can be used
for many different problems in machine learning. In particular, we can use NNs for
classification and regression problems such as determining if a planet is terrestrial
or not, or fitting terrain data. It is therefore of interest to measure the performance
and quality of NNs compared with other machine learning algorithms.

In this article, we will study the 2005 Taiwan credit card data set with the purpose
of classifying risky/non-risky credit card holders. For this, we use two methods:
logistic regression and artificial neural networks. We also use a NN to fit
the Franke function (see our earlier paper \citet{prosjekt1}). In both cases, we compare
our ANN to the respective methods: logistic and linear regression \citep{prosjekt1}.

Starting with section \ref{sec:theory}, we describe the data set and
the methods used in detail, as well as our chosen activation, cost, and score functions.
Moving on to section \ref{sec:results}, we present our most important results such as
tuning parameters and accuracy scores for each method. In section \ref{sec:discussion}
we discuss the method comparisons in more detail, and consider our results. Furthermore,
possible lacks and improvements are suggested, before concluding our paper in section
\ref{sec:conclusion}.
