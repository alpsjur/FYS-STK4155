\section{Introduction}
\label{sec:introduction}
In physics, as well as many other fields, differential equations (DE's) play a central role in analysis of a wide variety of problems. As such, it is important to use an efficient and accurate algorithm to produce reliable results. There exists many such algorithms of different orders, and each have their uses for different DE's. More complex DE's usually require higher order algorithms such as fourth order Runge-Kutta, while simple DE's can do with a first or second order algorithm such as Forward Euler (FE), or Centered Difference (CD), respectively. Higher order methods usually perform slower, and for complex coupled models, it is important to choose just the right algorithm for solving each sub-problem, as a slow method might significantly slow down model runs. It is therefore of interest if a general method for solving DE's exist which is efficient. A possible candidate are Neural Networks (NN's), which can approximate any function. In general, NN's for solving DE's could be useful if they offer a good accuracy/efficiency trade-off for many different types of equations, or if they outperform traditional methods. We will focus on the latter, and compare a traditional CD/FE-approach with a NN-approach. Research provided by \cite{lagaris1998artificial} and \cite{chiaramonte2013solving} suggest that one can indeed solve PDEs and ODEs with Neural Networks with high precision and several advantages,but that it comes at a computational cost. We will see if we when solving the Diffusion Equation, come to the same conclusion. Furthermore, we will also look at an immediate application of the NN-approach in finding extrema eigenpairs \citep{yi2004neural} and compare this with the standard approach.

The Partial Differential Equation (PDE) we will solve (Eq. \ref{eq:diffusionEQ}), as well as all background theory and methods, can be found in Section \ref{sec:theory}. Our most important results are showcased in Section \ref{sec:results}, and mainly consists of comparisons between different methods. In Section \ref{sec:discussion} we discuss the pros and cons of each method, as well as taking a deeper look at interesting results. Finally, in section \ref{sec:conclusion} we summarise the article, presenting the most important takeaways, as well as possible future uses.
