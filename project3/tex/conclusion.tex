\section{Conclusion}
\label{sec:conclusion}

We used a neural network to solve the diffusion equation, and compared this with a finite difference solution. Our findings show that a neural network can be used to solve a partial differential equation. The neural network is for the most part slower than the forward euler and centered difference schemes for a given MSE. However, for an MSE of around $10^{-8}$ they are around even. The neural network might be faster for higher precision, but this would require more computing power to confirm. Further tuning of the neural network could also be done to decrease the MSE further.

We then used a neural network to find the extreme eigenpairs of a real, symmetric matrix by solving an ordinary differential equation. Comparing with \texttt{numpy.linalg}, the difference in eigenvalues was in the order of $10^{-5}$, when the network was set to train until the  cost function went below $10^{-4}$. But the high precision made the network considerably slower than \texttt{numpy.linalg}, indicating that the latter method is favourable. In addition, the converging time and iterations seemed to be dependent on the matrix. Possibly, a better implementation of the network could deal with some of these factors.